\chapter*{APPENDICES}
\addcontentsline{toc}{chapter}{APPENDICES}

\begingroup\let\clearpage\relax
\chapter{Coding Guidelines}
\label{chap:Coding Conventions}

The Coding Guidelines are not written by me, they are a copy of our coding guidelines at work at Ruag. I got permission to include them in this document. 
\section{Coding Style}
\begin{itemize}
	\item In general qtcreator >= 3.3 default C++ (Qt built-in) code style
	\begin{itemize}
		\item	\url{http://wiki.qt.io/Coding-Conventions}
		\item	\url{http://wiki.qt.io/API_Design_Principles}
	\end{itemize}
	\item Class names: Start with capital letter, continue with Capital Letter for each word. SourceFilter.
	\item File Names: source\_filter.h source\_filter\_p.h source\_filter.cpp (as it is)
	\item Use \textbf{"abstract"} instead of "interface": AbstractDataSource
	\item Local Variables, \textbf{camelCase}: isEnabled
	\item Global Variables, prefix \textbf{with underscore}: \_isEnabled
	\item \textbf{No} variable \textbf{prefixes} like \sout{lisWidgets}
	\item Function Names, camelCase: setEnabled
	\item Indention: \textbf{4 spaces} instead of tab
	\item Doxygen: Class, Public Functions, Enums and Private Functions have a \textbf{doxygen comment in the header file}. Variables optional
	\item \textbf{Getter / Setter} should stick together in .h and .cpp
	\item Pointer symbol and ampersand belong to variable type (Foo* foo, Bar\& bar)
	\item \textbf{Signal / Slots} have no prefix \sout{slotStartExport}
	\item Try to have a \textbf{minimal list of includes}.
	\item Your Code shall produce \textbf{no compiler Warnings!}
\end{itemize}

\textbf{If/else statement}
\begin{lstlisting}
if (condition) {

} else {

}
\end{lstlisting}
\textbf{Switch case}
\begin{lstlisting}
switch (control) {
case value:

	break;
default:
	break;
}
\end{lstlisting}
\textbf{For loop}
\begin{lstlisting}
for (unsigned var = 0; var < total; var++) {

}
\end{lstlisting}
\textbf{Range based for loop}
\begin{lstlisting}
for (auto const& e : container) {

}
\end{lstlisting}
\textbf{While loop}
\begin{lstlisting}
while (condition) {

}
\end{lstlisting}
\textbf{const / ampersand placement}
\begin{lstlisting}
void Foo::printDebug(QString const& text)
{
    qDebug() << text;
}
\end{lstlisting}
\textbf{Connect}
\begin{lstlisting}
connect(instance, &Class::cameraChanged, ui->cameraBox, &QComboBox::setCurrentIndex);

connect(_controller, &Controller::updateViewport ,this, (void(QOpenGLWidget::*)())&QOpenGLWidget::update);

connect(ui->counterButton, &QAbstractButton::pressed, [=] () {
    int const currentValue = ui->counterButton->text().toInt();
    ui->counterButton->setText(QString::number(currentValue + 1));
});
\end{lstlisting}
\newpage
\section{Best Practices \& Tips}
\textbf{QSharedPointer}
Illustration of a Common Problem:
\begin{lstlisting}
int a;
QSharedPointer<int> pa (&a);
QSharedPointer<int> pb (&a);
pa!=pb //Both shared pointers will have a separate refcounting
\end{lstlisting}
You shall not pass a raw pointer to One Object more than once to a QSharedPointer (Constructor). 
If you have a second location where you need a QSharedPointer to the same Object then you have to Construct that SharedPointer with the Copy-Constructor.

Also: QSharedPointer<...>(this) is always wrong. Use: \url{http://doc.qt.io/qt-5/qenablesharedfromthis.html}

Useful Debugging Helper: \url{http://doc.qt.io/qt-5/qsharedpointer.html#optional-pointer-tracking}

\textbf{QList and QVector}
\begin{itemize}
	\item \textbf{Do not use `QList`s.} 
	
	They're officialy slow! `QVarLengthArray`, `QVector`, `std::vector` and `std::deque` are your friends.
	Why? QLists stores only pointer to elements, unless sizeof(T) < sizeof(void*) and you declared T to be a Q\_MOVABLE\_TYPE (with Q\_DECLARE\_TYPEINFO).
	More infos in the Slides from Oliver Gaffort \url{http://t.co/MEjWZhZ5L2}
	\item \textbf{Do not pass QLists and QVectors around as Pointers} 
	
	QList and QVector are implicitly shared. Passing them around as pointers only raises questions about the ownership, and does not improve performance in any Way.
	\item \textbf{Do not use range based for loops on Qt-Containers without paying extrem attention!}
	
	Using a range based for loop on a Qt-Container which is non-const will cause a deep copy to occour even if you declare the element to be constant. (Reason: QVector::begin() causes a detach). 
	Use `foreach` (=`QT\_FOREACH`) on qt containers and range-based for loops on std containers.
\end{itemize}
\textbf{Custom (Pointer) Types in QVariant}

If you have to store custom types in QVariant, store it with the correct type:
\begin{lstlisting}
Bla* foo = ...;

//Wrong!!!!
QVariant v = QVariant::fromValue(static_cast<void*>(foo));
Bla* bar = static_cast<Bla*>(v.value<void*>());

//Correct:
QVariant v = QVariant::fromValue(foo);
Bla* bar = v.value<Bla*>();
\end{lstlisting}

\textbf{But this means that I have to register T*\_ with as a Metatype?!}

Yes, in the most cases. 
But note that you don't have to use qRegisterMetaType\textless T*\textgreater(); because Q\_DECLARE\_METATYPE(T*) is enough.
\newpage
\textbf{Don't forget to free instances hidden in a QVariant}

Common problem:
\begin{lstlisting}
QVariant v = QVariant::fromValue(new Foo());
//... 
//Programm terminates
//Value stored in Variant was never freed -> MemoryLeak.
\end{lstlisting}

Either ensure that the Owner of the Object deletes it, or store a QSharedPointer<Foo> in the QVariant!












\endgroup