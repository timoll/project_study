\chapter{Introduction}
\label{chap:introduction}



% Eintr�ge im Verzeichnis erscheinen lassen ohne hier eine Referenz einzuf�gen
\nocite{kopka:band1}
\nocite{raichle:bibtex_programmierung}
\nocite{MiKTeX}
\nocite{KOMA}
\nocite{TeXnicCenter}
\nocite{Marti06}
\nocite{Erbsland08}
\nocite{juergens:einfuehrung}
\nocite{juergens:fortgeschritten}

\section{Analyzing Eye Tracking}
\label{sec:einleitung_aufbau}
Eye tracking is used to measure where someone is looking. It is used in a wide variety of applications such as marketing research, psychology, virtual reality and sports training. 

To enable high speed eye tracking for sports the HuCE developed an eye tracking system called the Gazelle Eye Tracker that is fast, portable and built for outdoor usage.   

For analyzing the recorded footage a player is required that can put an overlay over the video playback. This overlay needs to be synchronized with each frame of the video.



\section{Requirements}
\label{sec:introduction_contact}

There a few key requirements that can be categorized in optional and must have. This is done in \ref{tab:requirements}. 

\begin{table}[H]
	\centering
	\begin{tabular}{lcc} \toprule
		\textbf{Requirement} & \textbf{must} & \textbf{optional} \\ \midrule
		Play video & x &  \\ \midrule
		Video has overlay & x &  \\ \midrule
		Step frame for frame \\ forward and backward& x &  \\ \midrule
		Step overlay for overlay \\ forward and backward& x &  \\ \midrule
		Overlay and Frames are in sync & x &  \\ \midrule
		Display data of eye-cameras &  & x \\ \midrule
		Play at various playspeeds &  & x \\ \midrule
		Play overlays &  & x \\ \midrule
	\end{tabular}
	\caption{List of requirements }
	\label{tab:requirements}
\end{table}

